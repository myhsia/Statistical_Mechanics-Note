\documentclass[12pt, legalpaper]{article}
\usepackage[scheme = plain, linespread = 1.4, fontset = lxgw]{ctex}
\usepackage{amsmath, physics2}
\usephysicsmodule{ab, ab.braket, op.legacy}
\usepackage[margin = .8cm, paperheight = 24.28cm, paperwidth = 11.2cm]{geometry}
\usepackage{multirow}
\long\def \turnpage#1 {\par\noindent\rule{\dimeval{.5\linewidth-2.5\ccwd}}{.5pt}\hfill
  {\small \lower.25\ccwd\hbox{\fangsong 翻页(#1\%)}}\hfill
  \rule{\dimeval{.5\linewidth-2.5\ccwd}}{.5pt}\par}
\setlength \parindent {0pt}

\begin{document}

各位下午好. 我们组汇报的主题是\textbf{Bose-Hubbard 模型的数值计算与算法优化}.
我是汇报人余林涛,小组成员还有夏明宇和肖月两位同学.
我们的研究聚焦强关联量子多体系统的数值模拟,通过设计并对比三种量子蒙特卡洛
(MCMC)采样算法,实现采样效率的优化. 接下来将按
\begin{enumerate}
  \item 模型介绍
  \item SSE 级数展开
  \item 数值计算分析
  \item 模拟结果
  \item 核心结论
  \item 仿真方法
\end{enumerate}
六个部分讲解.

\turnpage{5}

\section{模型介绍}

\turnpage{7}

Bose-Hubbard 模型是描述无自旋玻色子在周期性晶格上相互作用的基础理论模型,
广泛应用于超流-莫特绝缘相变、高温超导等领域的研究. 其哈密顿量的完整形式如下
\begin{equation}
  \mathcal H = -t \sum_{\braket<i,j>} \hat b_i^\dagger \hat b_j
      + \frac12 U \sum_i \hat n_i (\hat n_i - 1) - \mu \sum_i \hat n_i.
\end{equation}
\turnpage{9}
其中
\begin{enumerate}
  \item 跳跃项推动系统进入超流相(粒子无耗散流动):
  玻色子以跃迁振幅 $t$ 在相邻格点间自由移动,趋向于让粒子在晶格上均匀分布.
  \turnpage{12}
  \item 在位排斥项推动系统进入莫特绝缘相(粒子被束缚在单个格点,无宏观流动):
  描述同一格点上两个玻色子的排斥作用($n_i = 0$ 或 $1$ 时该项为 $0$,$n_i \geq2$
  时随粒子数增长);排斥强度由 $U$ 控制,阻碍粒子在同一格点聚集;
  当 $U > 9$ 时,排斥作用完全压制跳跃作用,系统进入莫特绝缘相;
  $U$ 较小时($U < 3$),跳跃作用主导,系统处于超流相;
  \turnpage{14}
  \item 化学势项 $-\mu \sum_i \hat n_i$:描述外部场对玻色子的能量贡献,
  用于控制系统的平均粒子数密度.
\end{enumerate}

\turnpage{16}

\section{SSE 级数展开}

\turnpage{19}

由于哈密顿量的非对易性,直接计算量子多体系统的配分函数极其困难.
随机级数展开(SSE)是解决这一问题的核心方法,其本质是通过泰勒展开将指数算子转化
为“经典闭合图”的求和,实现量子问题的经典化转化.
在我们的模型中配分函数的 SSE 展开式:
\begin{equation}
  Z = \sum_{m=0}^\infty \frac{\beta^m}{m!}
      \sum_{\{i_1,\ldots,i_m\}} \sum_{\{b_1,\ldots,b_m\}}
      \prod_{k=1}^m \braket<i_k|-H_{b_k}|i_{k+1}>.
\end{equation}
\turnpage{21}
为直观展示展开过程,我们以 1D-4 格点,展开阶数为 $m = 3$ 为例:
这里的四个状态序列需满足“守恒+闭合”条件
\turnpage{23}
\begin{enumerate}
  \item $\ket|i_1> = \ket|2, 1, 1, 0>$ 为初始状态,总粒子数为 4;
  \item 第二个状态序列 $\ket|i_2> = \ket|1, 2, 1, 0>$:
  代表格点 $1$ 的 $1$ 个粒子跃迁至格点 $2$,总粒子数仍为 $4$,这就是守恒条件;
  \item 第三个状态序列 $\ket|i_3> = \ket|1, 2, 1, 0>$:
  在位排斥算子 $(U, 3)$ 作用于格点 3(粒子数为 1,
  $\hat n_3(\hat n_3 - 1) = 0$),化学势算子 $(\mu, 4)$ 作用于格点 4
  (粒子数为 0),均不改变粒子数状态;
  \item 第四个状态序列 $\ket|i_4> = \ket|2, 1, 1, 0> = \ket|i_1>$,
  可以看到它回归初始状态,这即是闭合条件.
\end{enumerate}

通过该实例可见,SSE 的核心价值是将抽象的量子配分函数,转化为
“算子序列+状态序列+顶点权重”的经典构型集合,
后续 MCMC 算法的本质就是对这些经典构型进行高效采样,从而计算物理量的统计平均值.

\turnpage{26}

\section{算法设计}

我们的核心目标是通过优化 MCMC 的跃迁矩阵,减少采样过程中的“样本相关性” ---
用密度积分自相关时间 $\tau_\text{int}(n)$ 量化,也就是说 $\tau_\text{int}(n)$
越短,样本独立性越强,采样效率越高. 接下来详细介绍三种算法的原理

\turnpage{28}

\subsection{方案 A:热浴更新}

这种方法的核心原理是基于权重比例来进行采样,直接将跃迁概率设定为归一化后的顶点权重,即 $\pi_j = w_j / \sum_{k=1}^4 w_k$
(4 种散射过程:反弹、直线、跳跃、转向);这种方法没有拒绝步骤,
始终按权重接受跃迁,实现起来比较简单,仅需对权重归一化即可.

\turnpage{30}

跃迁矩阵结构
\[
  T_A =
  \begin{bmatrix}
    \pi_1 & \pi_2 & \pi_3 & \pi_4 \\
    \pi_2 & \pi_1 & \pi_4 & \pi_3 \\
    \pi_3 & \pi_4 & \pi_1 & \pi_2 \\
    \pi_4 & \pi_3 & \pi_2 & \pi_1
  \end{bmatrix}
\]
其中对角元 $T_{A_{ii}}=\pi_i$,
代表“滞留概率” --- 粒子可停留在当前散射过程,无需转移;

\turnpage{33}

这种方法的缺点在于:
对角元非零导致严重的“状态滞留”,样本间相关性极强,采样效率低下.

\turnpage{35}

\subsection{方案 B:最小反弹算法}

这种方法是基于线性规划的约束优化,核心目标是最小化反弹过程的总概率,
即最小化跃迁矩阵的迹 $\Tr(T_B) = \sum_{i=1}^4 T_{B_{ii}}$;
通过两大核心约束来实
\begin{enumerate}
  \item 概率归一化约束:
  对任意行索引 $i$,$\sum_{j=1}^4 T_{B_{ij}}=1$(确保每一步跃迁的概率和为 1);
  \item 细致平衡约束:
  对任意 $i$, $j$,$w_i T_{B_{ij}} = w_j T_{B_{ji}}$(保证采样的无偏性);
\end{enumerate}

\turnpage{37}

通过一个实例验证(权重 $[0.2, 0.4, 0.3, 0.1]$).
求解线性规划后得到跃迁矩阵
\[
  T_B =
  \begin{bmatrix}
    0.1 & 0.5 & 0.3 & 0.1 \\
    0.5 & 0.1 & 0.1 & 0.3 \\
    0.3 & 0.1 & 0.1 & 0.5 \\
    0.1 & 0.3 & 0.5 & 0.1
  \end{bmatrix}.
\]
可见反弹过程的对角元 $T_{B_{11}} = 0.1$,较热浴更新的 $\pi_1 = 0.2$ 显著降低.

\turnpage{40}

有效减少无效转移.

\turnpage{42}

\subsection{方案 C:局部最优算法(Locally Optimal Algorithm)——定理驱动算法}

这种方法基于 Peskun 定理(Peskun’s Ordering Theorem),该定理指出:
在满足细致平衡的前提下,跃迁矩阵的“非对角元占比越高”,采样效率越高;基于此,
核心设计规则为仅保留权重最大的散射过程的滞留概率,其余过程禁止滞留
($T_{C_{ii}}=0, i \neq \text{max}$);

\turnpage{44}

跃迁矩阵结构(设 $\pi_1 \leq \pi_2 \leq \pi_3 \leq \pi_4$,
$\pi_4$ 为最大权重):
\[
  T_C =
  \begin{bmatrix}
    0 & \frac{\pi_2}{1-\pi_1} & \frac{\pi_3}{1-\pi_1} &
    \frac{\pi_4}{1-\pi_1} \\
    \frac{\pi_1}{1-\pi_2} & 0 & \frac{\pi_3}{1-\pi_2} &
    \frac{\pi_4}{1-\pi_2} \\
    \frac{\pi_1}{1-\pi_3} & \frac{\pi_2}{1-\pi_3} & 0 &
    \frac{\pi_4}{1-\pi_3} \\
    \frac{\pi_1}{1-\pi_4'} & \frac{\pi_2}{1-\pi_4'} &
    \frac{\pi_3}{1-\pi_4'} & \pi_4'
  \end{bmatrix}.
\]
其中 $\pi_4' = 1 - \sum_{j \neq 4} T_{C_{4j}}$(仅最大权重过程允许少量滞留),非对角元需重新归一化以满足概率和为 1;

\turnpage{47}

它的优势是:禁止低权重状态滞留,最大化样本独立性,适配低 $U$ 场景
(权重分布均衡).

\turnpage{49}

\section{计算结果分析}

我们的核心观测指标为 $\tau_{int}(n)$(自相关时间越小,效率越高),量化结果如下

\turnpage{51}

\subsection{整体趋势}

\turnpage{53}

方案 A 热浴更新的 $\tau_{int}(n)$ 始终是 B、C 的 $2-3$ 倍,
在所有 $U$ 区间均显著低效;

\turnpage{56}

然后,我们逐渐增大扫描次数的取值.
我们会发现

\turnpage{58}

随 $U$ 增大,SSE 构型空间的对角权重逐渐主导,方案 C 局部最优算法的
$\tau_{int}(n)$ 略有上升,方案 B 最小反弹算法保持比较稳定;

扫描次数 $\geq 4000$ 时,三种算法的结果均趋于稳定,无明显波动.

\turnpage{60}

\subsection{不同相区的性能对比}

\turnpage{63}

在这个表格中,

\turnpage{65}

我们可以看到
\begin{enumerate}
  \item 在低 $U$(权重均衡,超流相)区间:方案 C 局部最优算法最优,
  $\tau_{int}(n)$ 仅 $7-9$;
  \item 在高 $U$ 区间(对角权重主导,近莫特相):
  方案 B 最小反弹算法最优,$\tau_{int}(n)$ 仅 $15-20$;
  \item 传统的方案 A 热浴更新因状态滞留问题,在所有场景下均无优势.
\end{enumerate}

\turnpage{67}

\section{核心结论与展望}

基于以上详细的模型分析、算法设计与仿真验证,得出三点核心结论

\turnpage{70}

\begin{enumerate}
  \item 热浴更新算法(A)的低效具有必然性:
  其跃迁矩阵的非零对角元导致严重的状态滞留,样本相关性强,
  自相关时间是优化算法的 $2-3$ 倍,不适用于强关联量子多体系统的高效采样;
  \turnpage{72}
  \item 算法选择必须依赖模型参数:
  在低 $U$ 区间(超流相,权重分布均衡)优先选择局部最优算法,
  在高 $U$ 区间(近莫特相,对角权重主导)优先选择最小反弹算法,
  通过“场景适配”可以有效提升采样效率;
  \turnpage{74}
  \item 高效算法的统一设计原则为:使“非最大权重状态的对角元归零”,
  剩余自由度需根据权重分布特性进行适配.
\end{enumerate}

\turnpage{77}

\section{仿真方法}

\turnpage{79}

我们使用了此表列出的关键参数.

\turnpage{81}

并使用 SSE 将量子问题映射到 $(d + 1)$ 维的经典图像.

\turnpage{84}

计算的流程如表所示.

\turnpage{86}

接下来是我们使用的计算程序. 我们使用 \textsf{Python} 进行仿真.

\turnpage{88}

首先初始化类用于存储参数.

\turnpage{91}

然后定义计算顶点权重的函数.

\turnpage{93}

再定义计算转移矩阵的函数.

\turnpage{95}

这里用到了此矩阵的表达式.

\turnpage{98}

最后在运行仿真时,执行马尔可夫链扫描.

\turnpage{100}

以上就是我们小组的全部分享,感谢大家的耐心聆听!
如有任何疑问或交流需求,欢迎提出.

\end{document}