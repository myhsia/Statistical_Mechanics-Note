\chapter{Microregular Ensemble}

平衡态统计一般理论是系综理论. 适用范围:宏观多粒子系统.

系综:微正则系综(基本系综),正则系综,巨正则系综.
\begin{itemize}
  \item 微正则系综:$E$, $N$, $V$ 固定
  \item 正则系综:$T$, $N$, $V$ 固定
  \item 巨正则系综:$T$, $\mu$, $V$ 固定
\end{itemize}

\section{经典统计系综}

经典力学的微观状态:相空间中一个点 $(q,p)$ 满足正则运动方程
\begin{equation}
  \dot q_i = \pdv H{p_i},~ \dot p_i = -\pdv H{q_i}, ~ i = 1,\ \ldots\,,~s
\end{equation}
$\{(q_n(t), p_n(t))\}$
相轨道 $(\dot q_i(t), \dot p_i(t))$,轨道上任意一点 $\d\Omega = \prod_i \d q_i \d p_i$, $i = 1$, $\ldots\,$,~$s$.

设 $\Gamma$ 为给定宏观物理条件下所有可能的微观状态 $\tilde\rho \d\Omega$:
$\d\Omega$ 内的微观状态数,则 $\rho \d\Omega = \frac{\tilde\rho \d\Omega}\Gamma = \frac{\tilde\rho \d\Omega}{\int \tilde\rho \d\Omega}$ 是某微观状态出现在 $\d\Omega$ 内的几率,满足归一化 $\int \rho \d\Omega = 1$,$\rho$ 为几率密度.

任何物理可观测量 $O$ 是微观力学量 $O$ 的统计平均.
\begin{equation}
  \overline O = \int \d\Omega\rho O
\end{equation}
\begin{itemize}
  \item 系统处于某一微观状态 $\leftrightharpoons$ 处于该微观状态的系统
  \item 处于 $\d\Omega$ 中的系统是 $\tilde \rho \d\Omega$ 个 $\Gamma$ 个系统的集合称为一个统计系综.
  \item 系综是假想的和所研究系统性质完全相同的彼此独立、各自处于某一微观状态的大量系统的集合.
\end{itemize}

\section{系综所满足的方程:Liouville 定理}

\begin{theorem}[Liouville 定理]
  系综的几率密度 $\rho$ 在运动中不变,
  \[\odv \rho t = 0, ~ \odv{\tilde\rho}t = 0.\]
  代表点数守恒
\begin{equation}
  \pdv\rho t + \nabla \bm J_\rho = 0,~ \bm J_\rho = \rho \bm r,
  \nabla = \ab(\pdv*{}{q_i},\pdv*{}{p_i}),~ \bm v = \sum_i (\dot q_i, \dot p_i)
\end{equation}
\begin{equation}
  \odv{\tilde\rho}t = \pdv{\tilde\rho}t + \sum_i \bm r_i (\rho\bm r_i \bm r_i)
= \pdv{\tilde\rho}t + \sum_i \ab(\pdv{\tilde p}q_i\dot q_i + \pdv{\tilde p}p_i\dot p_i)
= -\tilde\rho \sum_i \ab\{\pdv H{q_i,p_j} - \pdv H{p_i,q_j}\} = 0
\end{equation}
最后得出 Liouville 方程
\begin{equation}
  \pdv{\tilde\rho}t + \{\tilde\rho,\rho\} = 0
\end{equation}
\end{theorem}

\section{量子统计系综}

\begin{itemize}
  \item 对量子力学系统,我们用波函数 $\psi_n$ 或态 $\ket|n>$ 来代替相空间的 $(q,p)$
  \item $A_n = \braket<n|\hat A|n>$
  \item 统计系综,考虑一系列的态 $\ket|1>$, $\ket|2>$, $\ldots\,$,~$\ket|n>$.
  \item 第 $n$ 个态有 $\tilde \rho_n$ 个简并度,即有 $\tilde\rho_n$ 个系统.
\end{itemize}

总系统数 $N = \sum_n \tilde \rho_n$

$\rho_n = \frac{\tilde\rho_N}{N}$ 处于第 $n$ 个态的几率

$\sum_n \rho_n = 1$, $\overline A = \braket<A> = \sum_n \rho_n A_n$

统计算符(密度矩阵) $\hat \rho = \sum_n \ket|n> \rho_n \bra<n|$

$\{\ket|i>\}$ 一套正交\footnote{即 $\delta\braket<i|j> = \delta_{ij}$}完备\footnote{即 $\sum_i \ketbra|i><i| = \mathbbm 1$: 对于 $\{\ket|0>m \ket|\uparrow>, \ket|\downarrow>, \ket|\uparrow\downarrow>\}$,存在 $\ketbra|0><0| + \ketbra|\uparrow><\uparrow| + \ketbra|\downarrow><\downarrow| + \ketbra|\uparrow\downarrow><\uparrow\downarrow| = \mathbbm 1$}基.

密度矩阵
\begin{equation}
  \rho_{ij} = \braket<i|\hat\rho|j> = \sum_n \braket<i|n> \rho_n \braket<n|j>
\end{equation}
\[
  A_{ij} = \braket<i|A|j>,~ \overline A = \sum_n \rho_n \braket<n|A|n> = \sum_{ij} \sum_n \rho_n \braket<n|j> \braket<j|A|i>\braket<i|n> = \sum_{ij} \rho_{ij}A_{ji} = \Tr (\hat \rho A)
\]
\[\Tr \sum_i \rho_{ii} = 1\]

$\hat\rho$, $\ket|n>$ Schr\"odinger eq
\begin{equation}
  \iu\pdv*{}t \ket|n> = \hat H\ket|n>
\end{equation}
\[
  \iu \pdv*{}t \hat\rho = \sum_n \ab[\ab(\iu\pdv*{}t\ket|n>)\rho_n\bra<n| - \ket|n>\rho_n\ab(-\iu\pdv*{}t\bra<n|)]
  = \sum_n H\ket|n> \rho_n\bra<n| - \ket|n>\rho_n\bra<n| H = H\hat\rho - \hat\rho H = [H,\hat\rho]
\]
Finally, we have
\begin{equation}
  \pdv*{}t\hat\rho + \iu[H,\hat p] = 0
\end{equation}
即 $\hat\rho$ 的 Heisenberg eq. of motion.

\section{微正则系综}

\begin{itemize}
  \item 经典微正则系综:$E$, $N$, $V$ 不变系综 --- 孤立系.
\end{itemize}
由 Liouville 定理
\[\odv\rho t = 0\]
若在平衡态物理量不随时间变化,就要求在相空间固定点,$\rho$ 不随时间变化,即必要条件 $\pdv \rho t = 0$.
$\Longrightarrow$ 在相轨道内 $\rho$ 为常数.

但 Liouville 定理和平衡态物理量不变不能保证不同轨道的 $\rho$ 相同.

微正则系综的基本假设
\begin{itemize}
  \item 当 $H(q,p) = E$ 时,$\rho$ 是常数,即相空间中的等能面.
  \item 当 $H(q,p) \neq E$ 时(存在集合 $\{p,q\}$),$\rho = 0$.
\end{itemize}
To summarize
\begin{equation}
  \rho =
  \begin{cases}
    C & E \leq H \leq E + \Delta E,\\
    0 & otherwise
  \end{cases}
\end{equation}
守恒条件 (Normalization of $\rho$)
\begin{equation}
  \lim_{\Delta \to 0} C \int_{\Delta E} \d\Omega = 1.
\end{equation}
The mean value
\begin{equation}
  \overline O = \lim_{\Delta E \to 0} C \int_{\Delta O \d \Omega}.
\end{equation}
量子微正则系综
\begin{equation}
  H(q,p) \longrightarrow E_n
\end{equation}
加入
\begin{enumerate}
  \item 粒子的全同性
  \item $\rho_n =
  \begin{cases}
    C, & E_n = E\\
    0, & E_n \neq E
  \end{cases}$,$n$ 为标记量子态的量子数.
  $\sum_{n(E_n=E)} \rho_n = C$ ($\sum_{n(E_n=E)} 1 = 1$).
\end{enumerate}
\[
  \mathcal N(E,V,N) = \sum_{n(E_n=E)} 1, ~ C = \frac{1}{\mathcal N(E,N,V)}
\]