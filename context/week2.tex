% !TeX root = ../main.tex

\section{Homework \#2 [2025-09-09]}

\begin{problem}
  对独立粒子体系,用排列组合公式对可区分粒子、玻色子和费米子在给定粒子数分布 $\{a_\alpha\}$
  下的量子状态数 $W(\{a_\alpha\})$.
\end{problem}
\begin{solution}\leavevmode
  \begin{enumext}
    \item 可区分粒子\\
    由于粒子可区分,能级 $\varepsilon_\alpha$ 有 $g_\alpha$ 个简并量子态.
    将 $N$ 个粒子分成若干组 $\{a_\alpha\}$,分配方式数为
    \[
      \frac{N!}{\prod_{\alpha} a_\alpha!}
    \]
    对能级 $\alpha$,每个粒子可占据 $g_\alpha$ 个态中的任意一个,
    因此有 $g_\alpha^{a_\alpha}$ 种占据方式. 总方式数为
    \[
      W = \frac{N!}{\prod_{\alpha} a_\alpha!} \times \prod_{\alpha} g_\alpha^{a_\alpha}
    = N! \prod_{\alpha} \frac{g_\alpha^{a_\alpha}}{a_\alpha!}
    \]
    \item 玻色子\\
    粒子全同,每个量子态占据粒子数不限.  
    对能级 $\alpha$:将 $a_\alpha$ 个全同粒子放入 $g_\alpha$ 个态,等价于 $a_\alpha$ 个粒子与 $g_\alpha - 1$ 个棒隔开不同态的排列数:
    \[
      \frac{(a_\alpha + g_\alpha - 1)!}{a_\alpha! \,(g_\alpha - 1)!}
    \]
    各能级独立,所以:
    \[
      W = \prod_{\alpha} \frac{(a_\alpha + g_\alpha - 1)!}{a_\alpha! \,(g_\alpha - 1)!}
    \]
    \item 费米子\\
    粒子全同,受泡利原理限制:每个量子态最多一个粒子,且 $a_\alpha \le g_\alpha$.  
    对能级 $\alpha$:从 $g_\alpha$ 个态中选择 $a_\alpha$ 个被占据的方式数为组合数:
    \[
      \frac{g_\alpha!}{a_\alpha! \,(g_\alpha - a_\alpha)!}
    \]
    各能级独立,所以:
    \[
    W = \prod_{\alpha} \frac{g_\alpha!}{a_\alpha! \,(g_\alpha - a_\alpha)!}
    \]
  \end{enumext}
\end{solution}

\begin{problem}
  用最可几分布求出上题相应的配分函数.
\end{problem}
\begin{solution}\leavevmode
  \begin{enumext}
    \item 可区分粒子(MB 统计)
由 $\ln W = \ln N! + \sum_\alpha [a_\alpha \ln g_\alpha - \ln a_\alpha!]$ 及约束变分得  
\[
a_\alpha = g_\alpha \upe^{-\alpha - \beta E_\alpha}
\]
代入 $\sum a_\alpha = N$ 得 $\upe^{-\alpha} = N / Z_1$,于是  
\[
a_\alpha = N \frac{g_\alpha \upe^{-\beta E_\alpha}}{Z_1}, \quad Z_N = Z_1^N
\]
(或 $Z_N = Z_1^N / N!$ 以修正吉布斯佯谬)

\item 玻色子(BE 统计)
由 $\ln W = \sum_\alpha [\ln(a_\alpha + g_\alpha - 1)! - \ln a_\alpha! - \ln(g_\alpha - 1)!]$ 变分得  
\[
a_\alpha = \frac{g_\alpha}{\upe^{\alpha + \beta E_\alpha} - 1}
\]
令 $\alpha = -\beta \mu$,则  
\[
a_\alpha = \frac{g_\alpha}{\upe^{\beta(E_\alpha - \mu)} - 1}, \quad
\Xi = \prod_\alpha \ab(1 - \upe^{-\beta(E_\alpha - \mu)})^{-g_\alpha}
\]

\item 费米子(FD 统计)
由 $\ln W = \sum_\alpha [\ln g_\alpha! - \ln a_\alpha! - \ln(g_\alpha - a_\alpha)!]$ 变分得  
\[
a_\alpha = \frac{g_\alpha}{\upe^{\alpha + \beta E_\alpha} + 1}
\]
令 $\alpha = -\beta \mu$,则  
\[
a_\alpha = \frac{g_\alpha}{\upe^{\beta(E_\alpha - \mu)} + 1}, \quad
\Xi = \prod_\alpha \ab(1 + \upe^{-\beta(E_\alpha - \mu)})^{g_\alpha}
\]
  \end{enumext}
\end{solution}

\begin{problem}
  一个二能级系统,$\epsilon_1 = -\epsilon$, $\epsilon_2 = \epsilon$, 
  且 $g_1 = g_2 = 1$. 设有 $N$ 个独立可区分粒子处于平衡态,求
  \begin{enumext}
    \item 温度 $T \to 0$ 时系统的熵.
    \item 若``粒子''是自旋 $\uparrow$, $\downarrow$ 两个态,
    则 $T \to 0$ 的熵值在此时的物理意义是什么?
  \end{enumext}
\end{problem}
\begin{solution}\leavevmode
  \begin{enumext}
    \item 
    单粒子配分函数 $Z_1 = \upe^{\beta\epsilon} + \upe^{-\beta\epsilon} = 2\cosh(\beta\epsilon)$,系统配分函数 $Z_N = Z_1^N$. 熵  
    \[
      S = Nk \ab[\ln(2\cosh(\beta\epsilon)) - \beta\epsilon \tanh(\beta\epsilon)]
    \]  
    当 $T \to 0$,$\beta\epsilon \to \infty$,$\tanh(\beta\epsilon) \to 1$,$\cosh(\beta\epsilon) \sim \frac12 \upe^{\beta\epsilon}$,  
    \[
    \ln(2\cosh(\beta\epsilon)) \to \beta\epsilon
    \quad\Rightarrow\quad
    S \to Nk [\beta\epsilon - \beta\epsilon] = 0
    \]  
    所以 $S(T\to 0) = 0$.
    \item
    若为自旋系统,$T\to 0$ 时所有自旋处于低能态(完全极化),系统处于唯一基态,微观状态数 $W=1$,熵为零,符合热力学第三定律.
  \end{enumext}
\end{solution}

\begin{problem}
  论证光子气体不发生玻色 -- 爱因斯坦凝聚.
\end{problem}
\begin{solution}
  光子气体化学势 $\mu = 0$ 且 $\epsilon_{\min} = 0$,故 $\mu$ 始终等于最低能级,不存在随温度降低而趋近于零的过程.
  同时光子数不守恒,总粒子数由平衡条件调节,无 BEC 所需的粒子数重新分布相变机制.  
  因此光子气体不发生玻色–爱因斯坦凝聚.
\end{solution}

\begin{problem}[林宗涵《热力学与统计物理》 7.5]
  计算爱因斯坦固体模型的熵.
\end{problem}
\begin{solution}
  爱因斯坦固体模型可看作近独立子系,每一个子系的 Maxwell-Boltzmann 分布函数为
  \[
    Z = \sum_{n=0}^\infty \upe^{-\beta\epsilon_n}
  = \sum_{n=0}^\infty \upe^{-\beta(n+1/2)\hbar\omega}
  = \frac{\upe^{-\beta\hbar\omega/2}}{1 - \upe^{-\beta\hbar\omega}}
  \]
  设原子总数为 $N$,则总振动自由度为 $3N$. 系统的熵为
  \[
    S = 3Nk\ab(\ln Z - \beta\pdv*{\ln Z}\beta)
      = 3Nk\ab[\frac{\hbar\omega/kT}{\upe^{\hbar\omega/kT} - 1} - \ln(1 - \upe^{\hbar\omega/kT})]
  \]
\end{solution}

\begin{problem}[林宗涵《热力学与统计物理》 7.7]
  自旋为 $\hbar/2$ 的粒子处于磁场 $\mathcal H$ 中,粒子的磁矩为 $\mu$,
  磁矩与磁场方向平行或反平行所相应的能量分别为 $-\mu\mathcal H$ 与 $\mu\mathcal H$.
  今设有 $N$ 个这样的定域粒子处于磁场 $\mathcal H$ 中,整个系统处于温度为 $T$ 的平衡态,
  粒子之间的相互作用很弱,可以忽略.
  \begin{enumext}
    \item 求子系统的配分函数 $Z$.
    \item 求系统的自由能 $F$,熵 $S$,内能 $\bar E$ 和热容 $C_\mathcal H$.
    \item 证明总磁矩的平均值为 $\bar{\mathcal M} = N\mu\tanh\ab(\frac{\mu\mathcal H}{kT})$.
    \item 证明在高温弱场下,亦即 $\frac{\mu\mathcal H}{kT} \ll 1$ 时:
    $\bar{\mathcal M} = \frac{N\mu^2}{kT}\mathcal H$;
    磁化率 $\chi = \pdv{(\bar{\mathcal M}/V)}{\mathcal H} = \frac{N\mu^2}{kT}$;
    在低温强场下,亦即 $\frac{\mu\mathcal H}{kT} \gg 1$ 时:$\bar{\mathcal M} = N\mu$; $\chi = 0$.
  \end{enumext}
\end{problem}
\begin{solution}\leavevmode
  \begin{enumext}
    \item 代入配分函数的定义得
    \[
      Z = \upe^{\beta\mu\mathcal H} + \upe^{-\beta\mu\mathcal H}
    = 2\cosh(\beta\mu\mathcal H)
    \]
    \item
    \begin{enumext}
      \item 自由能 $F = -NkT\ln Z = -NkT\ln(2\cosh(\beta\mu\mathcal H))$.
      \item 熵 $S = Nk\ab(\ln Z - \beta\pdv{\ln Z}\beta)
    = Nk\ab[\ln(2\cosh(\beta\mu\mathcal H) - \beta\mu\mathcal H\tanh(\beta\mu\mathcal H))]$.
      \item 内能 $\bar E = -N\pdv{\ln Z}\beta = -N\mu\mathcal H \tanh(\beta\mu\mathcal H)$.
      \item 热容 $C_\mathcal H = \ab(\pdv{\bar E}T)_\mathcal H = Nk\ab(\frac{\mu\mathcal H}{kT})^2 \ab\{1 - \tanh^2\ab(\frac{\mu\mathcal H}{kT})\}$.
    \end{enumext}
      \item 设原子总数为 $N$. 则处于平行与反平行的概率分别为
      \[
        P_1 = \frac NZ\upe^{\beta\mu\mathcal H}, \quad
        P_2 = \frac NZ\upe^{-\beta\mu\mathcal H}.
      \]
      则磁矩的期望值为
      \[
        \bar{\mathcal M} = \braket<\mu> = P_1\mu + P_2(-\mu) = N\mu\tanh\ab(\frac{\mu\mathcal H}{kT}).
      \]
      \item 由于以下极限
      \[
        \lim_{x\to 0} \tanh x = x, ~ \lim_{x\to\infty}\tanh x = 1,~
        \lim_{x\to 0} \cosh x = 1, ~ \lim_{x\to\infty}\cosh x = \infty,~
      \]
      所以在高温弱场、低温强场下
      \[
        \lim_{T\to\infty} \bar{\mathcal M} = \frac{N\mu^2}{kT}\mathcal H, \qq{and}
        \lim_{T\to0} \bar{\mathcal M} = N\mu,~
      \]
      磁导率的一般表达式
      \[
        \chi = \pdv{(\bar{\mathcal M}/V)}{\mathcal H}
      = \frac{N\mu^2}{kT}\frac1{\cosh^2(\mu\mathcal H/kT)}
      \]
      则在在高温弱场、低温强场下
      \[
        \lim_{T\to\infty} \chi = \frac{N\mu^2}{kT}, \qq{and}
        \lim_{T\to0} \chi = 0
      \]
  \end{enumext}
\end{solution}

\begin{problem}[林宗涵《热力学与统计物理》 7.15]
  粒子的态密度 $D(\epsilon)$ 定义为:$D(\epsilon) \d\epsilon$ 代表粒子的能量处于
  $\epsilon$ 与 $\epsilon + \d\epsilon$ 之间的量子态数(见原书 \S 7.15).
  这里指考虑粒子的平动自由度所对应的态密度.
  \begin{enumext}
    \item 设粒子的能谱(即能量与动量的关系)是非相对论性的,试分别对下列三种空间维数,
    求相应的态密度 $D(\epsilon)$:
    \begin{enumext}
      \item 粒子局限在体积为 $V$ 的三维空间内运动
      \[
        \epsilon = \frac1{2m} (p_x^2 + p_y^2 + p_z^2);
      \]
      \item 粒子局限在面积为 $A$ 的二维平面内运动
      \[
        \epsilon = \frac1{2m} (p_x^2 + p_y^2)
      \]
      \item 粒子局限在长度为 $L$ 的一维空间内运动
      \[
        \epsilon = \frac{p_x^2}{2m}
      \]
    \end{enumext}
    \item 设粒子的能谱是极端相对性的,即 $\epsilon = cp$, $p = |\bm p|$,
    试对空间维数分别为
    \begin{enumerate*}
      \item 三维
      \item 二维
      \item 一维
    \end{enumerate*}
    三种情况,求相应的 $D(\epsilon)$.
  \end{enumext}
\end{problem}
\begin{solution}\leavevmode
  \begin{enumext}
    \item 首先计算关系 $\odv p/\epsilon$
    \[
      \epsilon = \frac{p^2}{2m} \Rightarrow \odv p\epsilon = \frac mp
    \]
    三维、二维、一维情况下的态密度分别为
    \begin{align*}
      D_\text{3D}(\epsilon) &
    = \frac1{\d\epsilon}\int \frac{\d\omega}{h^3}
    = \frac1{\d\epsilon}\int\frac{\d x\d p_x \d y\d p_y \d z\d p_z}{h^3}
    = \frac{V}{h^3} 4\pi p^2\odv p\epsilon
    = \frac{2\pi V}{h^3}(2m)^{3/2}\epsilon^{1/2}\\
      D_\text{2D}(\epsilon) &
    = \frac1{\d\epsilon} \int\frac{\d x\d p_x \d y\d p_y}{h^3}
    = \frac{2\pi Am}{h^2}\\
      D_\text{1D}(\epsilon) &
    = \frac Lh \int 2\odv p\epsilon
    = \frac Lh(2m)^{1/2}\epsilon^{-1/2}
    \end{align*}
    \item $\epsilon = cp$ 时,$\odv p/\epsilon = \frac1c$. 只需将 (a) 中的 $\odv p/\epsilon$ 替换为新的 $\odv p/\epsilon$ 即可. 结果分别为
    \[
      D_\text{3D} = \frac{4\pi V}{(hc)^3}\epsilon^2,~
      D_\text{2D} = \frac{2\pi A}{(hc)^2}\epsilon,~
      D_\text{1D} = \frac{2L}{hc}.
    \]
  \end{enumext}
\end{solution}