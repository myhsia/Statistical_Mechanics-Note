\chapter{Review: Basic Concepts of Thermodynamics}

\section{Definitions}

\begin{definition}[Equilibrium State]
  在没有外界影响的条件下,物体部分的长时间不发生变化的状态.
\end{definition}
\begin{definition}[热平衡定律]
  A 与 B 平衡,B 与 C 平衡,则 A 与 C 平衡.
\end{definition}
\begin{definition}[Temperature]
  衡量物体间是否热平衡的物理量称为温度,一切互为热平衡的物体温度相等.
\end{definition}
\begin{definition}[温标]
  确定温度具体数值的规则叫温标
\end{definition}
\begin{definition}[物态方程]
  \begin{description}
    \item[几何变量] $V$, $A$, $L$
    \item[力学变量] $p$, $\sigma$, $F$
    \item[电磁变量] $\bm E$, $\bm p$, $\bm H$, $\bm M$
    \item[化学变量] $\mu$
  \end{description}
\end{definition}
\begin{equation}
  T = f(p, V, ...)
\end{equation}
\begin{definition}[内能]
  绝热过程(没有热量/能量交换的过程)中外界对物体做功只与初态和末态有关,
  初态和终态的内能差 $U_2 - U1 = W_a \qq{外界对物体的绝热功}$
\end{definition}

\begin{definition}[热力学第一定律]
  推广到非绝热过程,系统从外界吸热,$Q = U_2 - U_1 - W_0$(能量守恒).
\end{definition}

\begin{definition}[热容]
  \begin{equation}
    C_y = \odv{Q_y}T, \qq{$y$ 是一个不变的量}
  \end{equation}
  如果 $y = V$,称为定容;$y = p$,称为定压.

  比热 $C/V$
\end{definition}

\begin{definition}
  内能是态函数,$H = U + pV$,称为焓
  
  绝热过程中,$\Delta H = W_a$.

  等压过程中,$p$ 固定 $\Delta H = Q_p$.

  Entropy: 对可逆过程,态函数
  \begin{equation}
    \Delta S = S - S_0 = \int_\text{Initial State}^\text{Final State} \frac{\d Q}{T}
  \end{equation}
\end{definition}

\begin{definition}[热力学第二定律]
  \begin{equation}
    \Delta S \geq \int_{(i)}^{(f)} \frac{\d Q}T
  \end{equation}
\end{definition}

\begin{definition}[热力学基本方程]
  \begin{equation}
    \d U = T\d S = \sum_i F_n \d y_i
    p-V-T: \d U = T\d S - p\d V
  \end{equation}
  自由能:$F =  U - TS$
  $\d F = \d Y - \d(TS)$, 
  $\d F = -S\d T - p\d N$
\end{definition}

\begin{definition}[G.bbs 自由能: $G = F + pV$]
  \begin{equation}
    \d G = -S\d T + V\d p
  \end{equation}
  means 等温等压过程中,$G$ 又不增加.
\end{definition}

\section{均匀系(单相系)的平衡}

均匀系 $p-V-T$:
\begin{equation}
  \d U = T\d S - p\d V ~ (S,V)
\end{equation}
\begin{framed}
  \begin{equation}
    \d f = \ab(\pdv fx)_y\d x + \ab(\pdv fy)_x \d y
  \end{equation}
\end{framed}
\begin{equation}
  \ab(\pdv TV)_S = -\ab(PS)_V
\end{equation}
同理,对于焓
\begin{equation}
  \d H = T\d S + V\d P ~ (S,P)
\end{equation}
\begin{framed}
  \begin{equation}
    \ab(\pdv TP)_S = \ab(\pdv VS)_P
  \end{equation}
\end{framed}
\begin{gather}
  \d F = -S\d T - p\d V\\
  \d G = -S\d T + V\d P
\end{gather}

\begin{definition}[可测量热力学量]
  \begin{enumerate}
    \item $p$, $V$, $\ldots\,$,~$T$.
    \item 响应函数:压缩系数,膨胀系数,...
  \end{enumerate}
\end{definition}

\section{单元系的相变热力学}
\begin{itemize}
  \item 单相系 $\in$ 单元系
  \item 相变:整个单详细的性质发生了变化,从一个平衡态到另一个平衡态
  \item 系统处于某一个相中,就是系统处于热平衡,判据 $S = S_{\max} \Leftrightarrow$ 孤立系处于平衡态. $\delta S = \delta^2 S = 0$, $\delta U = \delta V = \delta N = 0$.
  \item $\delta S = 0$, $\delta^2S < 0$.
  \begin{itemize}
    \item $\delta^3 S = 0$ 是稳定的必要条件
    \item $\delta^4 S < 0 \to$ critical state
  \end{itemize}
\end{itemize}

\begin{enumerate}
  \item 自由能判据:$T$, $V$, $N$ 不变,$F = F_{\min}$
  \item Gibbs 自由能判据:$T$, $P$, $N$ are constants, $G = G_{\min}$.
\end{enumerate}

\begin{itemize}
  \item If the number of particles is changeable, then
  \begin{equation}
    \d U = T\d S - p\d V + (u + T_s + pV)\d N
  \end{equation}
  Here, $G/N = \mu$ is chemical potential.
  \item $\mu\d N$
  \begin{align}
    \mu & = \ab(\pdv UN)_{S,V} = \ab(\pdv HN)_{S,P} = \ab(\pdv FN)_{T,V} = \ab(\pdv GN_{T,P})\\
    \d \mu = -S\d T + \sigma \d P
  \end{align}
  \item $\Psi = F - \mu N = U - T_s - \mu N = F - G$ is called the giant potential (巨势).
\end{itemize}

由平衡判据,可以得到平衡条件.

如熵极大 $T_1 = T_2$(热平衡), $P_1 = P_2$(力学平衡), $\mu_1 = \mu_2$(化学平衡).

总粒子数不守恒 $\delta F = 0$, $P_1 = P_2$, $\mu_1 = \mu_2 = 0$.

由平衡判据,可以得到稳定条件

E.g.: 自由能极小
\[C_v > 0,~ K_T = \frac1V\ab(\pdv PV)_T > 0\]

\begin{itemize}
  \item Due to equilibrium conditions, we can obtain the \emph{phase diagram}.
  
  两相平衡,$\mu^1 = \mu^2$, $T_1 = T_2 = T$, $P_1 = P_2 = P$.
  \[\mu^1(T,P) = \mu^2(T,P), ~ T, P\text{平面上}\]
  
  Three-phase equilibrium: $\mu_1 = \mu_2 = \mu_3$.
\end{itemize}

\section{热力学第三定律}

\begin{definition}
  多元系的复相平衡和化学平衡
  ($T$, $P$, $N$, $\ldots\,$,~$N_k$) $\{N_i\}$ $\mu\d N \to \sum_i\mu_i \d N_i$
  $\mu_1 = \ab(\pdv\xi{N_i})_{T,P,\{N_{j\neq i}\}}$

  \[\int \d T - V\d q + \sum_i N_i \d \mu_i = 0\]
  $k + 1$ 是独立的.
\end{definition}

发生化学反应时,
\[\sum_{i=1}^k \nu_iA_i = 0\]
如
\[
  \begin{matrix}
    \ce{CO} & + & \frac12\ce{O2} & = & \ce{CO2}\\
    A_2 & & A_3 & & A_1\\
    \nu_2 = -1 & & \nu_3 = -\frac12 & & \nu_1 = 1
  \end{matrix}
\]
反应平衡条件
\begin{equation}
  \sum_i \nu_i\mu_i = 0
\end{equation}

一些经验关系
\begin{itemize}
  \item 等温等压条件下,反应向放热方向进行, $\Delta H < 0$.
  \item 等温等压化学反应,向着 $\Delta G$ 减小方向进行.
  \[\Delta G = \Delta H - TS \Rightarrow \lim_{T\to 0}(\Delta S)_T \to 0\]
  称为 Nernest Theorem.
\end{itemize}

\begin{definition}[热力学第三定律]
  绝对熵 $\lim_{T\to0} S = 0$:不可能通过有限步骤使物体冷却到绝对零度.
\end{definition}

\section{Linear Nonequilibrium Thermodynamics}

\begin{itemize}
  \item 能量守恒方程 -> 推广的热力学第一定律(每一小块质心运动考虑进去).
  \item 对小块,熵的微分方程成立.
  \item 第二定律:$\theta = \fdv St$ 表示小块熵产生率.
  \[\odv St = -\nabla \cdot \bm J_s + \theta. \text{$\bm J_s$ 为熵流密度}\]
  $\bm J_s = \frac{\bm J_q}T$, $\bm J_q$ 为热流,$\theta = \frac K{T^2}(\nabla T)^2 > 0$. $K$ 为热导率.
  \item $\pdv nt + \nabla \bm{J_n} = 0$
  \item 输运过程
\end{itemize}
Fourier: $\bm J_q = -K\nabla T$
Fick: $\bm J_n = -D_n\nabla T$, $\bm J_e = \sigma \bm E = -\sigma\nabla \phi$
