% !TeX root = ../main.tex

\section{Homework \#3 [2025-09-16]}

\begin{problem}
  $N$ 个单原子分子组成的理想气体,
  \[
    H = \sum_{i=1}^{3N} \frac{p_i^2}{2m}
  \]
  微观状态数的定义为
  \[
    \Omega(E) = \frac1{N!h^{3N}} \int_{E\leq H \leq E+\Delta E}
    \d q_1 \cdots \d q_{3N} \d p_1 \cdots \d p_{3N}
  \]
  证明
  \[
    \Omega(E) = \pdv{\Sigma(E)}E \Delta E
  \]
  其中 $\Sigma(E) = K \frac{V^N}{N!h^{3N}} (2mE)^{3N/2}$,
  $K = \frac{\pi^{3N/2}}{(3N/2)!}$.
\end{problem}
\begin{solution}\begin{proof}
  $N$ 个分子构成的 $3N$ 维 Euclidean 空间 (动量空间) 体积为
  \[
    V_p^{(3N)} = \int \prod_{i=1}^{3N} \d p_i
  = \frac{\pi^{3N/2}}{\Gamma(\frac32N + 1)} R^{3N}
  = \frac{\pi^{3N/2}}{(3N/2)!} R^{3N} = KR^{3N}
  \]
  其中 $R = \sqrt{2mE}$ 为动量空间半径. 则区间 $E \sim E + \Delta E$ 内的空间壳体积为
  \[
    \Delta V_p^{(3N)} = \pdv{V_p^{(3N)}}R \Delta R
  = 3NKR^{3N-1} \Delta R
  \xlongequal{\Delta R = m\Delta E/R} 3mNK(2mE)^{(3N-2)/2} \Delta E
  \]
  代入微观状态数的定义中
  \[
    \Omega(E) = \frac{3NmKV^N}{N!h^{3N}} (2mE)^{(3N-2)/2} \Delta E
  \]
  其中 $V^N = (\int \d^3q_i)^N$. 注意到
  \[
    \pdv{\Sigma(E)}E = \frac{3NmKV^N}{N!h^{3N}}(2mE)^{3N/2 - 1}
  \]
  于是证明了 $\Omega(E) = \pdv{\Sigma(E)}E \Delta E$.
  \end{proof}
\end{solution}

\begin{problem}
  一维谐振子
  \[
    H = \frac1{2m} p^2 + \frac k2 q^2
  \]
  证明
  \begin{enumext}
    \item 正则方程的解是
    \[
      q = A \cos(\omega t + \phi), \quad
      p = m\dot q = -m\omega A \sin(\omega t + \phi)
    \]
    $A$ 为振幅,$\omega = \sqrt{k/m}$ 是频率.
    \item 振子的能量为
    \[
      E = \frac12 m\omega^2A^2
    \]
    \item $(q,p)$ 在相空间的轨道是
    \[
      \frac{q^2}{\frac{2E}{m\omega^2}} + \frac{p^2}{2mE} = 1
    \]
    \item 求在能量区间 $E - \frac\Delta 2 \leq H \leq E + \frac\Delta2$,
    在相空间代表点的数目
    \[
      \int_{E-\Delta/2\leq H\leq E+\Delta/2} \d q \d p
    \]
  \end{enumext}
\end{problem}
\begin{solution}\leavevmode
  \begin{enumext}
    \item 由哈密顿正则方程
    \[
      \dot q = \frac pm, \quad \dot p = -kq
    \]
    将 $\dot q$ 再次对时间求导,得运动方程
    \[
      \ddot q = \frac{\dot p}m = -\frac km q,\quad
      \ddot q + \frac km q = 0
    \]
    则通解为
    \[
      q = A \cos(\omega t + \phi), \quad
      p = m\dot q = -m\omega A\sin(\omega t + \phi)
    \]
    其中 $\omega = \sqrt{k/m}$.
    \item 振子的能量为
    \[
      E = K + V = \frac{p^2}{2m} + \frac12 m\omega^2q^2
        = \frac12 m\dot q^2 + \frac12 m\omega^2 q^2 = \frac12 m\omega^2 A^2
    \]
    \item 将 $p$, $q$ 表达式合并
    \[
      \ab(\frac qA)^2 + \ab(\frac p{m\omega A})^2 = 1
    \]
    由 $E = \frac12 mA^2$ 得 $A^2 = 2E/m$. 代入得相空间轨道
    \[
      \frac{q^2}{\frac{2E}{m\omega^2}} + \frac{p^2}{2mE} = 1
    \]
    \item $(q,p)$ 在相空间的轨道为椭圆,其面积为
    \[
      A(E) = \pi ab = \pi \sqrt{\frac{2E}{m\omega^2}} \sqrt{2mE} = \frac{2\pi E}{\omega}
    \]
    则在能量区间 $E - \frac\Delta 2 \leq H \leq E + \frac\Delta2$,
    相空间代表点的数目即为能量区间的对应的相空间面积
    \[
      \int_{E-\Delta/2\leq H\leq E+\Delta/2} \d q \d p = A(E + \Delta/2) - A(E - \Delta/2) = \frac{2\pi\Delta}{\omega}
    \]
  \end{enumext}
\end{solution}
\pagebreak
\begin{problem}
  读 Pathria 书的 \S1.2,\S1.3,写一个阅读笔记.
\end{problem}
\paragraph{\S1.2 统计学与热力学之间的联系}
\begin{enumext}
  \item \textbf{系统描述与基本假设.}
  \begin{enumext}
    \item 两个系统 $A_1$ 和 $A_2$,分别处于平衡态,
    宏观态由 $(N_1, V_1, E_1)$ 和 $(N_2, V_2, E_2)$ 描述.
    \item 系统的微观状态数分别为 $\Omega_1(N_1, V_1, E_1)$, $\Omega_2(N_2, V_2, E_2)$.
    \item 复合系统 $A^{(0)} = A_1 + A_2$ 的总能量守恒
    $E^{(0)} = E_1 + E_2 = \text{Constant}$.
  \end{enumext}
  \item \textbf{复合系统的微观状态数.}
  $\Omega^{(0)}(E_1, E_2) = \Omega_1(E_1) \cdot \Omega_2(E_2)$
  \item \textbf{平衡条件与最概然状态.}
  平衡时,$\Omega^{(0)}$ 取最大值
  \[
    \frac{\partial \ln \Omega_1(E_1)}{\partial E_1}
  = \frac{\partial \ln \Omega_2(E_2)}{\partial E_2}
  \]
  定义 $\beta \equiv
    \ab( \frac{\partial \ln \Omega(N, V, E)}{\partial E} )_{N, V}$,
  则平衡条件为 $\beta_1 = \beta_2$.
  \item \textbf{熵与温度的联系.}
  热力学中存在关系
  \[
    \ab( \frac{\partial S}{\partial E} )_{N, V} = \frac{1}{T}
  \]
  对比统计定义 $S = k \ln \Omega$ 可得 $\beta = \frac{1}{k_BT}$,
  $k$ 为玻尔兹曼常数
\end{enumext}

\paragraph{\S1.3 统计学与热力学的进一步联系}
\leavevmode
\begin{enumext}
  \item \textbf{能量与体积交换.}
  若系统间可交换能量与体积,则平衡条件为
  \[
    \beta_1 = \beta_2 \qq{and} \eta_1 = \eta_2
  \]
  其中 $\eta \equiv \ab( \frac{\partial \ln \Omega}{\partial V} )_{N, E}$.
  \item \textbf{能量、体积与粒子数交换.}
  若还可交换粒子,则平衡条件为
  \[
    \beta_1 = \beta_2, \quad \eta_1 = \eta_2, \quad \zeta_1 = \zeta_2
  \]
  其中 $\zeta \equiv \ab( \frac{\partial \ln \Omega}{\partial N} )_{V, E}$.
  \item \textbf{与热力学量的对应.}
  由热力学基本关系
  \[
    \d E = T\d S - P\d V + \mu \d N
  \]
  可得 $\eta = \frac{P}{k_BT}$, $\zeta = -\frac{\mu}{k_BT}$.
  \item \textbf{统计热力学的核心公式.}
  \begin{enumext*}[columns = 5, label = \roman*., labelsep = 0pt, widest = 3]
    \item(2) 熵: $S(N, V, E) = k \ln \Omega(N, V, E)$.
    \item(3) 强度量: $\frac{1}{T} = \ab( \frac{\partial S}{\partial E} )_{N,V}$,
    $\frac{P}{T} = \ab( \frac{\partial S}{\partial V} )_{N,E}$,
    $-\frac{\mu}{T} = \ab( \frac{\partial S}{\partial N} )_{V,E}$.
  \end{enumext*}
\end{enumext}

\begin{problem}
  经典单原子分子理想气体,忽略气体内部自由度,用正则系综求内能,物态方程和熵.
\end{problem}
\begin{solution}
  考虑单粒子的 Hamiltonian,对于理想气体
  \[
    H(\bm p, \bm q) = \frac{\bm p^2}{2m}
  \]
  无势能,与 $\bm q$ 无关. 单粒子的配分函数为
  \[
    Z_0 = \frac1{h^3} \int_V \d^3\bm q \int_{\mathbb R^3} \upe^{-\beta H}
  = \frac{V}{h^3} \ab(\int_{-\infty}^\infty \upe^{-\beta p_i^2/(2m)} \d p_i)^3
  \]
  其中 $i = x$, $y$, $z$, $\beta = (k_BT)^{-1}$.
  这里利用了单粒子三个自由度之间的对称性,并引入量子相空间尺度 $h^3$ 作无量纲化.
  利用 Gaussian 积分
  \[
    \int_{-\infty}^\infty \upe^{-\beta p_i^2/(2m)} \d p_i = \sqrt{\frac{2\pi m}{\beta}}
  \]
  单粒子的配分函数可写做
  \[
    Z_0 = \frac{V}{h^3} \ab(\frac{2\pi m}{\beta})^{3/2} = \frac{V}{\lambda^3}
  \]
  其中 $\lambda \equiv \sqrt{\frac{\beta h^2}{2\pi m}}$ 为 de Broglie 波长.
  接下来考虑 $N$ 个不可区分粒子的配分函数,其可近似为
  \[
    Z_N = \frac{Z_0^N}{N!} = \frac1{N!} \ab(\frac V{\lambda^3})^N
  \]
  其中因子 $N!$ 为 Gibbs 修正,为了解不可区分性.
  \begin{enumext}
    \item 内能
    \[
      U = -\pdv*{\ln Z_N}\beta
      \xlongequal[\ln Z_N = N\ln Z_0 - \ln N!]{\text{Stirling identity}}
      -N \pdv*{\ln Z_0}\beta
    = \frac32 Nk_BT
    \]
    \item 状态方程.
    系统的自由能
    \[
      F= -k_BT \ln Z_N \xlongequal[\ln Z_N = N\ln Z_0 - \ln N!]{\text{Stirling identity}}
      -Nk_BT\ab[\ln\ab(\frac{V}{N\lambda^3}) + 1]
    \]
    由热力学关系 $p = -(\pdv F/V)_T$ 得
    \[
      p = k_BT\pdv*{\ln Z_N}V = \frac{nk_BT}{V}
    \]
    于是状态方程为
    \[
      pV = Nk_BT
    \]
    \item 熵. 由 $S = -(\pdv F/T)_V$ 得
    \[
      S = k_B\ln Z_N + k_BT\pdv*{\ln Z_N}T = Nk_B\ab[\ln\ab(\frac{V}{N\lambda^3} + 1)] + \frac32 k_BT
    = Nk\ab\{\ln\ab[\frac VN \ab(\frac{4\pi m}{\beta h^3})^{3/2}] + \frac52\}
    \]
    这里在 Stirling 公式的基础上做了一阶 Taylor 展开.
  \end{enumext}
\end{solution}

\begin{problem}[林宗涵《热力学与统计物理》 8.1]
  设有 $N$ 个粒子组成的系统处于平衡态,满足经典极限条件.
  \begin{enumext}
    \item 试由正则系统的几率分布导出系统微观能量处在 $E$ 与 $E + \d E$ 之间的几率
    $P(E) \d E$($P(E)$ 为正则系综按能量的几率分布).
    \item 证明使 $P(E)$ 取得极大值的能量满足方程
    \[
      \frac{\Sigma''(E)}{\Sigma(E)} = \beta
    \]
    其中 $\Sigma(E)$ 定义为
    \[
      \Sigma(E) = \frac1{N!h^s} \int_{H\leq E} \d\Omega
    \]
    $H = H(q_1, \cdots, q_s; p_1, \cdots, p_s)$ 为系统的Hamiltonian.
    \item 将上述结果用到单原子分子理想气体,证明
    \[
      E = \ab(\frac{3N}2 - 1) \frac1\beta \approx \frac32 Nk_BT.
    \]
    这个结果说明什么?
  \end{enumext}
\end{problem}
\begin{solution}\leavevmode
  \begin{enumext}
    \item 在正则系综中,系统处于某一微观状态 $i$ 的几率为:
    \[
      p_i = \frac1Z \upe^{-\beta E_i}, \qq{and}
      Z = \sum_i \upe^{-\beta E_i}
    \]
    系统的微观状态数函数
    \[
      \Sigma(E) = \frac1{N!h^{3N}} \int_{H\leq E} \d\Omega
    = \frac1{N!h^{3N}} \int_V \d^{3N} \bm q \int_{H\leq E} \d^{3N} \bm p
    = \frac{V^N}{N!h^{3N}} \frac{(2\pi mE)^{3N/2}}{\Gamma(3N/2 + 1)}
    \]
    则系统处在能量区间 $E \sim E + \d E$ 之间的概率为
    \[
      P(E) \d E = p_i (\Sigma(E + \d E) - \Sigma(E))
    = \frac1Z \upe^{-\beta E_i} \Sigma'(E) \d E
    \]
    \item $P(E)$ 最大时,$\odv{P(E)}E = 0$,即
    \[
      \odv{P(E)}E = -\frac\beta Z \upe^{-\beta E_i} \Sigma'(E) + \frac1Z \upe^{-\beta E_i} \Sigma''(E) = 0
    \]
    由此得 $\Sigma''(E)/\Sigma'(E) = \beta$.
    \item 展开 (b) 中的结论
    \[
      \frac{-\frac{3N}{2E}\Sigma(E) + \frac{3N}{2E} \frac{3N}{2E} \Sigma(E)}
        {\frac{3N}{2E}\Sigma(E)}
    = \frac{3N - 2}{2E} = \beta
    \]
    由此得最概然能量
    \[
      E = \frac1\beta \ab(\frac32N - 1)
      \xlongequal{N\gg1} \frac32Nk_BT
    \]
    表明使 $P(E)$ 取极大值的能量即平均能量,体现了统计物理中的大数定律.
  \end{enumext}
\end{solution}

\begin{problem}[林宗涵《热力学与统计物理》 8.2]
  有两种不同分子组成的混合理想气体,处于平衡态.
  设该气体满足经典极限条件;且可把分子当作质点(即忽略其内部运动自由度).
  试用正则系统求该气体的 $p$, $\bar E$, $S$, $\mu_i$ ($i = 1$, $2$).
\end{problem}
\begin{solution}
  设第一种与第二种分子数分别为 $N_1$, $N_2$, 微观能量分别为 $\epsilon_i$, $\epsilon_j$.
  则混合理想气体的微观总能量为
  \[
    E = E_1 + E_2 = \sum_{i=1}^{N_1} \epsilon_i + \sum_{i=1}^{N_1} \epsilon_j
  \]
  由 \textbf{\textsf{Problem 3.4}}: 单粒子的配分函数,则两种分子的配分函数分别为
  \[
    Z^{(1,2)} = \frac{Z_0^{(1,2)}}{N_{1,2}!}, \quad
    Z_0^{(1,2)} = \frac V{h^3} \ab(\frac{2\pi m_{1,2}}{\beta})^{3/2}
  \]
  则系统的配分函数为
  \[
    Z_{N_1, N_2} = Z_{N_1} Z_{N_2}
  = N_1 \ln Z^{(1)} - \ln N_1! + N_2 \ln Z^{(2)} - \ln N_2!
  \]
  则气体的参数为
  \begin{align*}
    p & = \frac1\beta \pdv*{\ln Z_{N_1, N_2}}V
  = \frac{(N_1 + N_2) k_BT}V,\\
    \bar E & = -\pdv*{\ln Z_{N_1, N_2}}\beta = (N_1 + N_2) \frac32 k_BT,\\
    S & = k(\ln Z_{N_1, N_2} - \beta \pdv*{\ln Z_{N_1, N_2}}\beta)
  = \sum_{i=1,2} N_i k_B \ab\{\frac32\ln T + \ln \frac V{N_i} +
                \ab[\frac52 + \frac32 \ln\ab(\frac{2\pi m_i k_B}{h^2})]\},\\
    \mu_i & = \ab(\pdv F{N_i})_{T,V} = \ab\bigg(\pdv{\bar E - TS}{N_i})_{T,V}
  = -k_BT \ab[\frac32\ln T + \ln \frac{V}{N_i}
  + \frac32 \ln\ab(\frac{2\pi m_i k_B}{h^2})]
  \end{align*}
\end{solution}

\begin{problem}[林宗涵《热力学与统计物理》 8.3]
  有一极端相对论性的理想气体,粒子的能谱为 $\epsilon = cp$
  ($p = |\bm p|$,$c$ 为光速),并满足非简并条件.
  设粒子的内部运动自由度可以忽略(即可将粒子看成质点).
  试用正则系综求该气体的 $p$, $\bar E$, $S$, $\mu$, $C_v$, $C_p$.
\end{problem}
\begin{solution}
  此时单粒子的配分函数为
  \[
    Z_0 = \frac1{h^2} \int_V \d^3\bm q \int_{\mathbb R^2} \d^3 \bm q \upe^{-\beta\epsilon}
  = \frac{V}{h^3} \int_0^\infty \upe^{-\beta cp} 4\pi p^2 \d p
  = \frac{8\pi V}{(hc)^3} (k_BT)^3
  \]
  系统的配分函数仍然成立
  \[
    Z_N = \frac{Z_0}{N!} = \frac{V^N T^{3N}}{N!} \ab[\frac{8\pi k_B^3}{(hc)^3}]^N
  \]
  则该系统的参数分别为
  \begin{align*}
    p & = -\ab(\pdv FV)_{T,N} = -\ab(\pdv{-k_BT\ln Z_N}V)_{T,N} = \frac{Nk_BT}{V},
    \quad
    \bar E = -T^2 \pdv*{(F/T)}T = 3Nk_BT,\\
    S & = -\ab(\pdv FT)_{V,N}
    = Nk_B\ab\{3\ln T + \ln \frac VN + \ab[4 + \ln\frac{8\pi k_B^3}{(hc)^3}]\},\\
    \mu & = -k_BT\ab\{3\ln T + \ln\frac VN + \ln\ab[\frac{8\pi k_B^3}{(hc)^3}]\},
    \quad
    C_v = \ab(\pdv{\bar E}T)_V = 3Nk_B, \quad
    C_p = \ab(\pdv HT)_p = 4Nk_B
  \end{align*}
  其中焓 $H \equiv \bar E + pV = 4Nk_BT$.
\end{solution}

\begin{problem}[林宗涵《热力学与统计物理》 8.6]
  设被吸附在液体表面上的分子形成一种二维气体,分子之间相互作用为两两作用的短程力,
  且只与两分子的质心距离有关. 试根据正则系综,证明在第二位力系数的近似下,该气体的物态方程为
  \[
    pA = Nk_BT\ab(1 + \frac{B_2}{A})
  \]
  其中 $A$ 为液面的面积,$B_2$ 由下式给出
  \[
    B_2 = -\frac N2 \int (\upe^{-\phi(r)/k_BT} - 1) 2\pi r \d r
  \]
\end{problem}
\begin{solution}
  气体的 Hamiltonian 为
  \[
    H = K + \Phi = \sum_{i=1}^N \frac{\bm p_i^2}{2m} + \sum_{i<j} \phi_{ij}
  \]
  其中 $U = \sum_{i<j} \phi(r_{ij})$
  对二维气体,配分函数为
  \[
    Z_N = \frac1{N!h^{2N}} \int \upe^{-\beta H} \d^{2N} \bm q \d^{2N} \bm p
  \]
  单个粒子的动量积分为
  \[
    \frac1{h^2} \int \exp\ab(-\frac{\beta p^2}{2m}) \d^2\bm p = \frac{2\pi mk_BT}{h^2} = \lambda_T^{-2}
  \]
  所以气体的配分函数为
  \[
    Z_N = \frac1{N!\lambda_T^{2N}}
          \int_{A^N} \upe^{-\beta\sum_{i<j}\phi_{ij}} \prod_{i=1}^N \d^2 \bm q_i
        = \frac1{N!\lambda_T^{2N}} Q_N
  \]
  其中 $Q_N$ 为位形积分. 使用 Mayer 函数 $f_{ij} = \upe^{-\beta\phi_{ij}} - 1$,
  位形积分在第二位力系数近似下可展开为
  \[
    Q_N \approx \int_{A^N} \ab(1 + \sum_{i<j} f_{ij}) \prod_{i=1}^N \d^2 \bm q_i
  = A^N \ab(1 + \frac{N^2}{2A} \int f_{12} 2\pi r \d r_1)
  \]
  其中 $f_{12} = f(\bm q_1 - \bm q_2)$, $\d^2\bm r_1 = 2\pi r_1 \d r_1$.
  令 $B_2 = -\frac N2 \int f_{12} \d^2\bm r_1$,则位形积分的对数可写做
  \[
    \ln Q_N = N \ln A + \ln\ab(1 - \frac NA B_2) \approx N \ln A - \frac NA B_2
  \]
  对二维气体,正则系综压强为
  \[
    p = \frac1\beta \pdv*{\ln Z_N}A \approx Nk_BT \ab(1 + \frac{B_2}{A})
  \]
\end{solution}

\begin{problem}[林宗涵《热力学与统计物理》 8.7]
  物质磁性的起源是纯量子力学性质的,这一点可以从玻尔-范列文(Bohr-van Leeuwen)定理看出.
  该定理可以表述为:遵从经典力学和经典统计力学的系统的磁化率严格等于零.
  \begin{remark}
    由公式 $\chi = \ab(\pdv{\mathcal M}{\mathcal H})_{T,V}$,
    $\mathcal M = -\ab(\pdv F{\mathcal H})_{T,V}$ 及 $F = -k_BT\ln Z_N$,
    只需证明正则系综的配分函数 $Z_N$ 与磁场 $\mathcal H$ 无关即可.
    设矢势为 $\bm A$(磁场由 $\bm A$ 定出),处于磁场中的 $N$ 个带电粒子系统的微观总能量
    (即系统的 Hamiltonian)可以表为
    \[
      E = \sum_{i=1}^N \frac1{2m} \ab(\bm p_i + \frac{e_i}{c} \bm A(\bm r_i))^2
    + \Phi(\bm r_1, \ldots, \bm r_N),
    \]
    其中 $\Phi$ 代表粒子之间的相互作用能.
    由正则系统出发,在满足经典极限条件下,证明 $Z_N$ 与 $\bm A$ 无关.
  \end{remark}
\end{problem}
\begin{solution}
  正则系统的配分函数为
  \[
    Z_N = \frac1{N!h^{3N}}\ab\{
      \int_{V^N} \upe^{-\beta\phi(\bm q_1, \cdots, \bm q_N)}
      \prod_{i=1}^N \d^3 \bm q_N\} \ab\{
      \int_{\mathbb R^{3N}}
        \exp\ab[-\beta \sum_i\ab(\bm p_i + \frac{e_i}{c} \bm A(\bm q_i))^2/2m]
        \prod_{i=1}^N \d^3 \bm p_i
    \}
  \]
  做动量积分的变量变换,令
  \[
    \bm p_i' = \bm p_i + \frac{e_i}{c} \bm A(\bm r_i)
  \]
  由多重积分变换
  \[
    \prod_{i=1}^N \d^3 \bm p_i = |J| \prod_{i=1}^N \d^3 \bm p_i'
  \]
  其中 Jacobian 为
  \[
    J = \frac
      {\partial(p_{1_x}, p_{1_y}, p_{1_z}, \ldots, p_{N_x}, p_{N_y}, p_{N_z})}
      {\partial(p_{1_x}', p_{1_y}', p_{1_z}', \ldots, p_{N_x}', p_{N_y}', p_{N_z}')}
  \]
  由于 $\pdv{p_i}/{\bm A(\bm r_i)} = 0$,所以 $J = 1$,$Z_N$ 与 $\bm A$ 无关.
\end{solution}

\begin{problem}
  用巨正则系综计算单原子理想气体的热力学函数.
\end{problem}
\begin{solution}
单粒子配分函数
\[
  Z_1 = \frac{V}{\lambda^3}, \quad \lambda = \frac{h}{\sqrt{2\pi m k_B T}}
\]  
由此得巨配分函数和巨势
\[
  \Xi = \exp\ab(\upe^{\beta\mu} Z_1)
= \exp\ab(\upe^{\beta\mu} \frac{V}{\lambda^3}), \quad
  \Omega = -k_B T \ln \Xi = -k_B T \upe^{\beta\mu} \frac{V}{\lambda^3}
\]
粒子数
\[
  \braket<N> = -\ab(\pdv\Omega\mu)_{T,V} = \upe^{\beta\mu} \frac{V}{\lambda^3}
  \quad\Longrightarrow\quad
  \upe^{\beta\mu} = \frac{\braket<N> \lambda^3}{V}
\]  
内能
\[
  U = \sum_{\bm p} \varepsilon_{\bm p} \braket<n_{\bm p}>
= \upe^{\beta\mu} \frac{V}{h^3} \int \frac{p^2}{2m} \upe^{-\beta p^2/(2m)} \d^3p
= \frac{3}{2\beta} \upe^{\beta\mu} \frac{V}{\lambda^3}
= \frac32 \braket<N> k_B T
\]
压强  
\[
  P = -\frac{\Omega}{V} = k_B T \upe^{\beta\mu} \frac{1}{\lambda^3}
    = \frac{\braket<N> k_B T}{V} \quad\Longrightarrow\quad
  PV = \braket<N> k_B T
\]  
由 $\Omega = U - TS - \mu \braket<N>$ 得熵
\[
  S = \frac{U - \mu\braket<N> - \Omega}{T}
    = \braket<N> k_B \ab[\frac52 - \ln(n \lambda^3)],
  \quad n = \frac{\braket<N>}{V}
\]
\end{solution}

\begin{problem}[林宗涵《热力学与统计物理》 8.9]
  试用巨正则系综求解题 \textbf{\textsf{Problem 3.7}},并于正则系综的结果比较.
\end{problem}
\begin{solution}
  由 \textbf{\textsf{Problem 3.7}} 中单粒子的配分函数
  $Z = \frac{8\pi V}{(hc)^3} \beta^{-3}$
  得巨正则系综函数
  \[
    \Xi = \sum_{N=0}^\infty \frac{(\upe^{-\alpha} Z)^N}{N!}
  = \exp(\upe^{-\alpha} Z), \quad
    \ln \Xi = \upe^{-\alpha} Z = \frac{8\pi V}{(hc)^3} \beta{-3}
  \]
  利用巨正则系综求解系统的参数为
  \begin{align*}
    \bar N & = -\pdv*{\ln\Xi}\alpha = \upe^{-\alpha} Z, \quad
    \mu = -k_BT \ln \frac ZN = -k_BT\ab[3\ln T + \ln\frac VN + \ln\ab[\frac{8\pi k^3}{(hc)^3}]],\\
    \bar E & = -\pdv*{\ln\Xi}p = 3\bar Nk_BT, \quad
    p = \frac1\beta \pdv*{\ln\Xi}V = \frac{\bar Nk_BT}{V} = \frac{\bar E}{3V},\\
    S & = k_B\ab(\ln\Xi - \alpha\pdv*{\ln X}\alpha - \beta\pdv*{\ln\Xi}\beta)
  = \bar Nk\ab\{3\ln T + \ln\frac{V}{\bar N} + \ab[4 + \ln\ab(\frac{8\pi k_B^3}{(hc)^3})]\},\\
    C_V & = \ab(\pdv*{\bar E}T)_V = 3Nk_B,\quad
    C_p = \ab(\pdv HT)_p = 4Nk_B
  \end{align*}
  结果与 \textbf{\textsf{Problem 3.7}} 一致.
\end{solution}

\begin{problem}[林宗涵《热力学与统计物理》 8.10]
  证明熵的下列公式.
  \begin{enumext}
    \item 对正则系综,$S = -k \sum_s \rho_s \ln \rho_s$,
    其中 $\rho_s = \frac1{Z_N} \upe^{-\beta E_s}$ 为正则系综的几率分布.
    \item 对巨正则系综,$S = -k \sum_N \sum_s \rho_{N_s} \ln \rho_{N_s}$,
    其中 $\rho_{N_s} = \frac1\Xi \upe^{-aN-\beta E_s}$ 为巨正则系综的几率分布.
  \end{enumext}
\end{problem}
\begin{solution}\leavevmode
  \begin{enumext}
    \item \begin{proof}
    由熵的定义出发
    \[
      S = -\pdv FT = -\pdv{-kT \ln Z}T
    = k\ln Z + \frac{\braket<E>}{T}
    \]
    其中 $Z = \sum_s \upe^{-\beta E_s}$, $\beta = (kT)^{-1}$,
    $\braket<E> = \frac1Z \sum_s E_s \upe^{-\beta E_s}$.
    将正则系综概率分布 $\rho_s = \frac1Z \upe^{-\beta E_s}$
    代入题目中正则系综中 $S$ 的右式
    \[
      -k \sum_s \rho_s \ln \rho_s = -k \sum_s \rho_s \ln\ab(\frac1Z \upe^{-\beta E_s})
      = k\ln Z + k\beta\braket<E>
    \]
    结果和 $S = -\pdv F/T$ 的表达式一致.
    \end{proof}
    \item \begin{proof}
      类似的,从巨势 $\Omega \equiv -kT \ln \Xi$ 出发,熵的表达式为
      \[
        S = -\pdv\Omega T = k \ln\Xi + \frac{\braket<E> - \mu\braket<N>}{T}
      \]
      其中巨正则系综几率分布 $\Xi = \sum_{N,s} \exp[\beta(\mu N - E_s)]$.
      从熵的定义出发可得
      \[
        S = -k \sum_N \sum_s \rho_{N_s} \ln \rho_{N_s}
      = -k \sum_{N,s} \rho_{N_s}(-\ln\Xi + \beta \mu N - \beta E_s)
      = k\ln\Xi + k\beta \mu\braket<N> - k\beta\braket<E>
      \]
      结果和 $S = -\pdv \Omega/T$ 的表达式一致.
    \end{proof}
  \end{enumext}
\end{solution}

\begin{problem}[林宗涵《热力学与统计物理》 8.12]
  设有一 $N$ 个相互作用可以忽略的粒子(可看成质点)组成的系统,
  在满足经典极限的条件下,巨正则系综的几率分布为
  \[
    \rho_N(q_1, \ldots, p_{3N}) \d\Omega_N = \frac1{\Xi N! h^{3N}}
    \upe^{-\alpha N-\beta E_N(q_1, \ldots, p_{3N})} \d\Omega_N
  \]
  \begin{enumext}
    \item 试证明巨正则系综的总粒子数是 $N$ 的几率为
    \[
      P(N) = \frac1\Xi \upe^{-\alpha N} Z_N,
    \]
    其中 $Z_N$ 是总粒子数为 $N$ 时的正则系综配分函数.
    \item 证明使 $P(N)$ 取极大的总粒子数满足下面的关系
    \[
      \alpha = \pdv{\ln Z_N}N.
    \]
    (证明时,直接求 $\ln P(N)$ 的极大更方便.)
    \item 上式进一步可化为
    \[
      N = \upe^{-a} Z
    \]
    其中 $Z$ 为单粒子的配分函数,即 $Z = \frac V{h^3} \ab(\frac{2\pi m}{\beta})^{3/2}$.
    上述结果说明什么?
  \end{enumext}
\end{problem}
\begin{solution}\leavevmode
  \begin{enumext}
    \item 巨正则系综的总粒子数是 $N$ 的几率可写做微观态的几率对相空间的积分
    \[
      P(N) = \int \rho_N(q_1, \ldots, q_{3N}; p_1, \ldots, p_{3N}) \d\Omega_N
    = \frac1{\Xi N! h^{3N}} \upe^{-\alpha N} \int \upe^{-\beta E_N} \d \Omega_N
    \]
    由于 $N$ 个粒子的正则系综配分函数为
    \[
      Z_N = \frac{1}{N!h^{3N}} \int \upe^{-\beta E_N} \d\Omega_N
    \]
    所以可得 $P(N) = \frac1\Xi \upe^{-\alpha N} Z_N$.
    \item 在 $P(N)$ 取极大时,$\pdv PN = 0$.
    将 (a) 中的表达式取对数并对 $N$ 求导得
    \[
      \pdv{\ln P(N)}N = -\alpha + \pdv{\ln Z_N}N = 0
    \]
    于是得 $\alpha = \pdv{\ln Z_N}/N$.
    \item 考虑 $N$ 个可忽略相互作用的粒子,系统的配分函数为
    \[
      Z_N = \frac{Z_0}{N!}
    \]
    其中 $Z_0 = \frac V{h^3} \ab(\frac{2\pi m}{\beta})^{3/2}$.
    对系统的配分函数取对数并对 $N$ 求导得
    \[
      \pdv{\ln Z_N}N = \ln Z - \ln N = \ln \frac ZN = \alpha
    \]
    由此得 $N = \upe^{-\alpha} Z \bar N$.
    即使 $P(N)$ 取极大值的 $N$ 就是平均值 $\bar N$.
  \end{enumext}
\end{solution}