\whatsnote_provide_module:n { typeset }

\RequirePackage{amssymb, bbm, bm, cancel, extarrows,
                mathtools, nicematrix, xfrac}
\RequirePackage[e]{esvect}
\let \vec \vv
\DeclareRobustCommand \cev [1]{{\mathpalette\do@cev{#1}}}
\newcommand \do@cev [2]{
  \vbox{\offinterlineskip
    \ialign{##\cr
      \hidewidth\reflectbox{$\m@th#1\vec{}\mkern4mu$}\hidewidth\cr
      $\m@th#1#2$\cr
    }
  }
}
\sys_if_engine_xetex:T { \PassOptionsToPackage { no-math } { fontspec } }
\RequirePackage{fixdif, derivative}
\newdif \D {\mathsf D}
\newdif \upe {\mathrm e}
\newdif \iu {\mathrm i \mkern1mu}
\RequirePackage{siunitx}
\DeclareSIUnit \angstrom {\text \AA}
\let \sinh \relax \let \cosh \relax \let \tanh \relax \let \coth \relax
\DeclareMathOperator \sinh  {sh}
\DeclareMathOperator \cosh  {ch}
\DeclareMathOperator \tanh  {th}
\DeclareMathOperator \coth  {cth}
\DeclareMathOperator \sgn   {sgn}
\DeclareMathOperator \cotan {\qopname\relax o{ctg}}
\DeclareMathOperator \im    {im}
\DeclareMathOperator \adj   {adj}
\DeclareRobustCommand \tran{^{\mkern-1.5mu\mathsf T}}
\DeclareRobustCommand \identity {\ensuremath{\mathbbm 1}}
\RequirePackage[version = 4]{mhchem}
\RequirePackage{physics2}
\usephysicsmodule{ab, ab.braket, diagmat, xmat, qtext.legacy, op.legacy}
\DeclareRobustCommand \bra
  {\@ifstar{\phy@d@lx{br.m}{br.a}}{\phy@d@lx{br.m}{br.a}*}}
\DeclareRobustCommand \ket
  {\@ifstar{\phy@d@lx{kt.m}{kt.a}}{\phy@d@lx{kt.m}{kt.a}*}}
\DeclareRobustCommand \braket
  {\@ifstar{\phy@d@lx{bk.m}{bk.a}}{\phy@d@lx{bk.m}{bk.a}*}}
\DeclareRobustCommand \ketbra
  {\@ifstar{\phy@d@lx{kb.m}{kb.a}}{\phy@d@lx{kb.m}{kb.a}*}}
\DeclareDocumentCommand \diagmat { s O{} m }
  {
    \IfBooleanTF {#1}
      { \__phy_diagmat_type:nnn {  small } {#2} {#3} }
      { \__phy_diagmat_type:nnn {        } {#2} {#3} }
  }
\DeclareDocumentCommand \pdiagmat { s O{} m }
  {
    \IfBooleanTF {#1}
      { \__phy_diagmat_type:nnn { psmall } {#2} {#3} }
      { \__phy_diagmat_type:nnn { p      } {#2} {#3} }
  }
\DeclareDocumentCommand \bdiagmat { s O{} m }
  {
    \IfBooleanTF {#1}
      { \__phy_diagmat_type:nnn { bsmall } {#2} {#3} }
      { \__phy_diagmat_type:nnn { b      } {#2} {#3} }
  }
\DeclareDocumentCommand \Bdiagmat { s O{} m }
  {
    \IfBooleanTF {#1}
      { \__phy_diagmat_type:nnn { Bsmall } {#2} {#3} }
      { \__phy_diagmat_type:nnn { B      } {#2} {#3} }
  }
\DeclareDocumentCommand \vdiagmat { s O{} m }
  {
    \IfBooleanTF {#1}
      { \__phy_diagmat_type:nnn { vsmall } {#2} {#3} }
      { \__phy_diagmat_type:nnn { v      } {#2} {#3} }
  }
\DeclareDocumentCommand \Vdiagmat { s O{} m }
  {
    \IfBooleanTF {#1}
      { \__phy_diagmat_type:nnn { Vsmall } {#2} {#3} }
      { \__phy_diagmat_type:nnn { V      } {#2} {#3} }
  }
\RequirePackage{hyperref}
\hypersetup{colorlinks, citecolor = teal}
\let\oldHyPsd@CatcodeWarning\HyPsd@CatcodeWarning
\renewcommand \HyPsd@CatcodeWarning [1]
  {\ifnum\pdfstrcmp{#1}{math shift}=0 \else \oldHyPsd@CatcodeWarning{#1} \fi}
\pdfstringdefDisableCommands{
  \def\intop{∫} \def\ilimits@{lim}
  \def\infty{∞} \let\HyPsd@CatcodeWarning\@gobble
}
\NewDocumentCommand \mathemph { O{} m } {\ifmmode
  \colorbox{Crimson!15}{\vphantom{$\displaystyle #1$}$\displaystyle #2$}\else
  \colorbox{Crimson!15}{\vphantom{#1}#2}\fi}
\color_set:nnn { eqcolor } { HTML } { 2F4F4F }
^^A \def \dollardollar@begin { \def \eqcolor {\color{eqcolor}} \eqcolor $$ }
^^A \def \dollardollar@end   { $$ \color_select:n {black} }
\DeclareRobustCommand \[
  {
    \scan_stop:
    \if_mode_math:
      \@badmath
    \else:
      \if_mode_vertical:
        \nointerlineskip
        \makebox[.6\linewidth]{ }
      \fi:
    \dollardollar@begin^^A \color_select:n { eqcolor }
    \fi:
  }
\DeclareRobustCommand \]
  {
    \scan_stop:
    \if_mode_math:
      \if_mode_inner:
        \@badmath
      \else:
        \dollardollar@end
      \fi:
    \else:
      \@badmath
    \fi:
    \ignorespaces
  }

\file_input_stop: